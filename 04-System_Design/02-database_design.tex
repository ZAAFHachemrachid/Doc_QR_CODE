\section{Database Design}
\begin{infobox}
This section details the PostgreSQL database design, including the schema, relationships, and optimization strategies.
\end{infobox}

\subsection{Entity Relationship Diagram}
\textcolor{TextBlack}{
    The database schema consists of the following main entities:
    \begin{itemize}
        \item \reference{Users (Students, Teachers, Administrators)}
        \item \reference{Courses}
        \item \reference{Attendance Records}
        \item \reference{QR Code Sessions}
        \item \reference{Academic Terms}
    \end{itemize}
}

\subsection{Database Schema}
\begin{notebox}
\textcolor{TextBlack}{
    Key tables and their relationships:

    \begin{verbatim}
    -- Users Table
    CREATE TABLE users (
        user_id SERIAL PRIMARY KEY,
        username VARCHAR(50) UNIQUE,
        role VARCHAR(20),
        email VARCHAR(100),
        created_at TIMESTAMP DEFAULT CURRENT_TIMESTAMP
    );

    -- Courses Table
    CREATE TABLE courses (
        course_id SERIAL PRIMARY KEY,
        course_code VARCHAR(20),
        course_name VARCHAR(100),
        teacher_id INTEGER REFERENCES users(user_id)
    );

    -- Attendance Records Table
    CREATE TABLE attendance_records (
        record_id SERIAL PRIMARY KEY,
        student_id INTEGER REFERENCES users(user_id),
        session_id INTEGER REFERENCES qr_sessions(session_id),
        timestamp TIMESTAMP,
        location_data JSONB
    );
    \end{verbatim}
}
\end{notebox}

\subsection{Optimization Strategies}
\textcolor{TextBlack}{
    The following strategies are implemented for optimal performance:
    \begin{itemize}
        \item \method{Indexed queries for frequent operations}
        \item \method{Partitioned tables for attendance records}
        \item \method{Materialized views for reporting}
        \item \method{JSON compression for location data}
        \item \method{Regular VACUUM and analysis}
    \end{itemize}
}
