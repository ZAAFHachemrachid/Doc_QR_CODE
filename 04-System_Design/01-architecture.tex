\section{System Architecture}
\begin{infobox}
This section describes the overall architecture of the QR Code Attendance System, including the technology stack and component interactions.
\end{infobox}

\subsection{High-Level Architecture}
\textcolor{TextBlack}{
    The system follows a three-tier architecture:
    \begin{itemize}
        \item \implementation{Mobile Application (Flutter)} - For students to scan QR codes
        \item \implementation{Backend Server (REST API)} - For business logic and data processing
        \item \implementation{Database (PostgreSQL)} - For data persistence and management
    \end{itemize}
}

\begin{figure}[h]
    \centering
    % Add your architecture diagram here
    \caption{High-Level System Architecture}
\end{figure}

\subsection{Technology Stack Selection}
\textcolor{TextBlack}{
    The following technologies were chosen for their specific advantages:

    \subsubsection{Frontend - Flutter}
    \begin{itemize}
        \item \result{Cross-platform development capability}
        \item \result{Rich UI components and smooth animations}
        \item \result{Built-in camera and QR code scanning libraries}
        \item \result{Strong security features for mobile applications}
        \item \result{Excellent performance and native compilation}
    \end{itemize}

    \subsubsection{Database - PostgreSQL}
    \begin{itemize}
        \item \result{Robust ACID compliance for data integrity}
        \item \result{Advanced security features and role management}
        \item \result{Excellent performance with large datasets}
        \item \result{JSON support for flexible data structures}
        \item \result{Strong community support and documentation}
    \end{itemize}
}
